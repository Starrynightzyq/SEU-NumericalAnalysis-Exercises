\chapter{第二章代码}

q2-1.py

\begin{lstlisting}[style = python]
import sys
import sympy

def func_1_expr():
    """
    x*sin(x) - 1 = 0
    """
    x = sympy.symbols('x')
    return x*sympy.sin(x)-1.0

def func_2_expr():
    """
    x^2 - 5 = 0
    """
    x = sympy.symbols('x')
    return x**2.0 - 5.0

def func_3_expr():
    """
    x^3 -3x + 2 = 0
    """
    x = sympy.symbols('x')
    return x**3.0 - 3.0*x +2

class _func():
    """
    计算一元函数值及导数值
    """
    def __init__(self, expr, eff=15):
        """
        初始化计算表达式
        expr@表达式
        eff@有效数字位数
        """
        self._expr = expr()
        self._eff = eff
        self._x = list(self._expr.free_symbols)[0]
        self._diff_expr = sympy.diff(self._expr, self._x)

    def value(self, x):
        """
        计算 func 值
        """
        expr = self._expr
        return expr.subs('x', x).evalf(self._eff)

    def diff_value(self, x):
        """
        计算导数值
        """
        expr = self._diff_expr
        return expr.subs('x', x).evalf(self._eff)

          

def TrailValue(expr, a, b, e):
    """
    试值法
    f@函数
    a@区间下限
    b@区间上限
    e@容忍误差限
    """
    f = _func(expr)
    fa_0 = f.value(a)
    fb_0 = f.value(b)
    res = 0
    count = 0
    if abs(fa_0) < e:
        res = a
    elif abs(fb_0) < e:
        res = b
    elif sympy.sign(fa_0) == sympy.sign(fb_0):
        print('f(a) and f(b) 同号')
        sys.exit()
    else:
        while True:
            count = count + 1
            fa = f.value(a)
            fb = f.value(b)
            c = b - ((b-a)/(fb - fa))*fb
            fc = f.value(c)

            # 更新有根区间
            if sympy.sign(fa) == sympy.sign(fc):
                a = c
            else:
                b = c

            # 判断计算结束
            if abs(f.value(c)) < e:
                res = c
                break

    return res, count

def Newton(expr, a, b, e):
    """
    牛顿法与二分法结合
    f@函数
    a@区间下限
    b@区间上限
    e@容忍误差限
    """
    f = _func(expr)
    fa_0 = f.value(a)
    fb_0 = f.value(b)
    res = 0
    count = 0
    if abs(fa_0) < e:
        res = a
    elif abs(fb_0) < e:
        res = b
    elif sympy.sign(fa_0) == sympy.sign(fb_0):
        print('f(a) and f(b) 同号')
        sys.exit()
    else:
        x = a
        while True:
            count = count + 1
            c = x - f.value(x)/f.diff_value(x)

            # Newton与二分法结合,找下一个点
            if (c > a) and (c < b):
                x = c
            else:
                x = a+(b-a)/2

            # 更新有根区间
            if sympy.sign(f.value(a)) == sympy.sign(f.value(x)):
                a = x
            else:
                b = x

            # 判断计算结束
            if abs(f.value(x)) < e:
                res = x
                break

    return res, count
        
def main():
    """
    main
    """
    res_t, count_t = TrailValue(func_1_expr, 1, 2, 1.0/sympy.Pow(10, 5))
    print('试值法 func 1 根: %.5f, 迭代次数: %d' % (res_t, count_t))

    res_n, count_n = Newton(func_1_expr, 1, 2, 1.0/sympy.Pow(10, 5))
    print('牛顿法 func 1 根: %.5f, 迭代次数: %d' % (res_n, count_n))

    res_t, count_t = TrailValue(func_2_expr, 2, 3, 1.0/sympy.Pow(10, 5))
    print('试值法 func 2 根: %.5f, 迭代次数: %d' % (res_t, count_t))

    res_n, count_n = Newton(func_2_expr, 2, 3, 1.0/sympy.Pow(10, 5))
    print('牛顿法 func 2 根: %.5f, 迭代次数: %d' % (res_n, count_n))

    res_t, count_t = TrailValue(func_3_expr, -2.5, -1.5, 1.0/sympy.Pow(10, 5))
    print('试值法 func 3 根: %.5f, 迭代次数: %d' % (res_t, count_t))

    res_n, count_n = Newton(func_3_expr, -2.5, -1.5, 1.0/sympy.Pow(10, 5))
    print('牛顿法 func 3 根: %.5f, 迭代次数: %d' % (res_n, count_n))

if __name__ == "__main__":
    main()
\end{lstlisting}